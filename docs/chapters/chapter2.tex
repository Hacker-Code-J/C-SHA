\chapter{SHA Family}

\begin{table}[h]
	\centering
	\begin{tabularx}{\textwidth}{@{\extracolsep{\fill}}|X|X|X|X|c|}
		\hline
		\textbf{Algorithm} & \textbf{Year} & \textbf{Developer} & \textbf{Design} & \textbf{Status} \\ \hline
		SHA-0              & 1993          & NSA                & MD+ARX          & Broken          \\ \hline
		SHA-1              & 1995          & NSA                & MD+ARX          & Broken          \\ \hline
		SHA-2              & 2001          & NSA                & MD+ARX          &                 \\ \hline
		SHA-3              & 2015          & Industry           & Sponge          &                 \\ \hline
	\end{tabularx}
	\caption{SHA Algorithm Versions}
	\label{tab:sha-versions}
\end{table}

\section{SHA-1}
\begin{tikzpicture}
	% Define styles
	\tikzset{
		blueblock/.style={rectangle, draw, fill=blue!20, text width=2.5em, text centered, minimum height=2.5em},
		cyanblock/.style={rectangle, draw, fill=cyan!20, text width=2.5em, text centered, minimum height=2.5em},
		redblock/.style={rectangle, draw, fill=red!20, text width=2.5em, text centered, minimum height=2.5em},
		greenblock/.style={rectangle, draw, fill=green!20, text width=2.5em, text centered, minimum height=2.5em},
		line/.style={draw, -Latex},
		xor/.style={circle, draw, minimum size=0cm, inner sep=0pt}
	}
	% Nodes
	\node[greenblock] (iv) {IV};
	\node[blueblock, right=2cm of iv] (cf0) {CF};
	\node[blueblock, right=2cm of cf0] (cf1) {CF};
	\node[blueblock, right=2cm of cf1] (cf2) {...};
	\node[blueblock, right=2cm of cf2] (cf3) {CF};
	\node[redblock, right=2cm of cf3] (hash) {Hash};
	
	\node[cyanblock, above=2cm of cf0] (m0) {$M[0]$};
	\node[cyanblock, above=2cm of cf1] (m1) {$M[1]$};
	\node[cyanblock, above=2cm of cf3] (mn) {$M[n]$};
	
	\draw[line] (iv) -- (cf0) node[midway, above] {160-bit};
	\draw[line] (cf0) -- (cf1) node[midway, above] {160-bit};
	\draw[line] (cf1) -- (cf2) node[midway, above] {160-bit};
	\draw[line] (cf2) -- (cf3) node[midway, above] {160-bit};
	\draw[line] (cf3) -- (hash) node[midway, above] {160-bit};
	
	\draw[line] (m0) -- (cf0) node[midway, left] {512-bit};
	\draw[line] (m1) -- (cf1) node[midway, left] {512-bit};
	\draw[line] (mn) -- (cf3) node[midway, left] {512-bit};
	
%	\node[below=1cm of p1] (xor1) {\Huge $\oplus$};
%	\node[below=1cm of p2] (xor2) {\Huge $\oplus$};
%	\node[below=1cm of p3] (xor3) {\Huge $\oplus$};
%	\node[below=1cm of p4] (xor4) {\Huge $\oplus$};
%	
%	\node[block, below=1cm of xor1] (e1) {$E_K$};
%	\node[block, below=1cm of xor2] (e2) {$E_K$};
%	\node[block, below=1cm of xor3] (e3) {...};
%	\node[block, below=1cm of xor4] (e4) {$E_K$};
%	
%	\node[block, below=1cm of e1] (c1) {$C_1$};
%	\node[block, below=1cm of e2] (c2) {$C_2$};
%	\node[block, below=1cm of e3] (c3) {...};
%	\node[block, below=1cm of e4] (c4) {$C_n$};
	
	% Arrows
%	\draw[line] (iv.south) to[bend right=45] (xor1.west);
%	\draw[line] (p1) -- (xor1);
%	\draw[line] (xor1) -- (e1);
%	\draw[line] (e1) -- (c1);
%	
%	\draw[line] (c1.east) to[bend right=45] ($(c1)!.5!(xor2)$) to[bend left=45] (xor2.west);
%	\draw[line] (p2) -- (xor2);
%	\draw[line] (xor2) -- (e2);
%	\draw[line] (e2) -- (c2);
%	
%	\draw[line] (c2.east) to[bend right=45] ($(c2)!.5!(xor3)$) to[bend left=45] (xor3.west);
%	\draw[line] (p3) -- (xor3);
%	\draw[line] (xor3) -- (e3);
%	\draw[line] (e3) -- (c3);
%	
%	\draw[line] (c3.east) to[bend right=45] ($(c3)!.5!(xor4)$) to[bend left=45] (xor4.west);
%	\draw[line] (p4) -- (xor4);
%	\draw[line] (xor4) -- (e4);
%	\draw[line] (e4) -- (c4);
\end{tikzpicture}


\begin{lstlisting}[style=C, caption={Key Expansion in C (General ver.)},captionpos=t]
// Define the SHA1 message digest structure
typedef struct {
	uint32_t state[5];
	uint32_t count[2];
	uint8_t buffer[64];
} sha1_t;

// Initialize the SHA1 message digest with a given seed
void sha1_init(sha1_t *sha1, uint32_t seed) {
	memset(sha1->state, 0, sizeof(sha1->state));
	sha1->count[0] = seed;
	sha1->count[1] = (seed >> 8) | ((seed & 0xff) << 24);
}

// Update the SHA1 message digest with a given block of data
void sha1_update(sha1_t *sha1, const void *data, size_t len) {
	while (len >= 64) {
		uint32_t words[16];
		for (int i = 0; i < 16; i++) {
			words[i] = ((uint32_t *)data)[i];
		}
		sha1->state[0] += words[0];
		sha1->state[1] += words[1];
		sha1->state[2] += words[2];
		sha1->state[3] += words[3];
		sha1->state[4] += words[4];
		for (int i = 0; i < 64; i++) {
			sha1->buffer[i] ^= ((uint8_t *)&words[i])[i & 3];
		}
		len -= 64;
		data += 64;
	}
}

// Finalize the SHA1 message digest and return the resulting hash
void sha1_final(sha1_t *sha1, uint8_t *hash) {
	sha1->state[0] = (sha1->state[0] & 0xff000000) | ((sha1->state[0] >> 24) & 0x00ffffff);
	sha1->state[1] = (sha1->state[1] & 0x00ffffff) | ((sha1->state[1] << 8) & 0xff000000);
	sha1->state[2] = (sha1->state[2] & 0x00ffffff) | ((sha1->state[2] << 16) & 0xff000000);
	sha1->state[3] = (sha1->state[3] & 0x00ffffff) | ((sha1->state[3] << 24) & 0xff000000);
	sha1->count[0] += 64;
	for (int i = 0; i < 5; i++) {
		hash[i] = (uint8_t)(sha1->state[i] >> 24);
		hash[i + 4] = (uint8_t)(sha1->state[i] >> 16);
		hash[i + 8] = (uint8_t)(sha1->state[i] >> 8);
		hash[i + 12] = (uint8_t)sha1->state[i];
	}
}
\end{lstlisting}
